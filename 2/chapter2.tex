%DO NOT MESS AROUND WITH THE CODE ON THIS PAGE UNLESS YOU %REALLY KNOW WHAT YOU ARE DOING
\chapter{Theory} \label{Theory}

\section{ Sound attenuation in water } \label{ Sound attenuation in water }

\noindent An acoustic signal underwater experiences attenuation due to spreading and absorption. The acoustic energy of a sound wave propagating in the ocean is partly absorbed, i.e. the energy is transformed into heat and partly lost due to sound scattering by in homogeneities. On the basis of extensive laboratory and field experiments the following empirical formulae for attenuation coefficient in sea water have been derived. Sound absorption in sea water is a function of frequency, temperature, salinity, pH, and pressure. When the acoustic wave propagates in sea water, absorption loss occurs, which is caused by a part of the energy changing into the heat owing to the viscous friction of the water molecule, aside from the spreading loss. The absorption loss is represented as $\alpha$ r, where  $\alpha$ is the coefficient in dB/Km  and r is the transmission distance.

\subsection{ Thorp formula } \label{ Thorp formula }

\noindent The absorption coefficient for frequency range 100 Hz to 3 kHz can be expressed empirically using Thorps formula which defines $\alpha_w$ [dB/m] as a function of f [kHz].

\[ \alpha_w  = \frac{0.11 f^2}{1 + f^2} + \frac{44 f^2}{4100 + f^2} \] 

\subsection{ Schulkin and Marsh formula } \label{ Schulkin and Marsh formula }

\noindent The absorption coefficient for frequency range 3 kHz to 500 kHz can be expressed empirically using Schulkin and Marsh model which defines  $\alpha_w$ [dB/m] as a function of f [kHz].

\[ \alpha_w  = 8.686.10^3 (\frac{S A f_t  f^2}{f_t ^2 + f^2} + \frac{B f^2}{f_t})(1 - 6.54.10^{-4} P) \] 

\noindent where  \(A = 2.34.10^{-6}, B = 3.38.10^{-6}\), S in [ppt], f in [kHz], relaxation frequency \(f_t = 21.9.10^\frac{6-1520}{T+273}\) with T in [$^{\circ}$C] for 0 $\leq T \leq$ 30 and the hydrostatic pressure is determined by \( P = 1.01 (1 + z 0.1)\) in [$kg/cm^3 = at$]

\subsection{ Francois and Garrison Formula } \label{ Francois and Garrison Formula }

\noindent  The absorption coefficient for frequency range 100 Hz to 1 MHz can be expressed empirically using Francois and Garrison model which defines  $\alpha_w$ [dB/m] as a function of f [kHz].

\[ \alpha_w  = \frac{ A _1 P_1 f_1 f^2}{f_t1^2 + f^2} + \frac{ A _2 P_2 f_2 f^2}{f_t2^2 + f^2}  + A_3 P_3 f^2 \] 

\noindent The first term gives the sound absorption due to the Boric Acid and second term gives the sound absorption due to the magnesium sulfate. The contribution of sound absorption due to these chemical ingredients has been found to be small. The third term represents the sound absorption due to pure water. The pressure dependency of above equation is shown by P1, P2 and P3 constants .Frequency dependency is given by f1 and f2 which are the relaxation frequencies of Boric Acid and Magnesium sulfate. f is the frequency of sound. The constants A1, A2 and A3 shown are not purely constants but it has been experimentally proved that their values vary with the water properties, like temperature, salinity and pH of water. The total coefficient of absorption of sea water is calculated by considering separately the absorption due to boric acid, magnesium sulphate and pure water. 

\noindent \(A_1 = \frac{8.686}{c} 10^{0.78 ph - 5}, f_1 = 2.8 \sqrt{\frac{S}{35}}10^{4 - \frac{1245}{T+273}}\)

\noindent \( P_1 = 1, c = 1412 +3.21 T + 1.19 S + 0.0167 Z_m\)

\noindent \(A_2 = 21.44\frac{S}{c}(1 + 0.0025 T),  f_2 = \frac{8.17.10^{\frac{8-1990}{T+273}}}{1+0.0018(S-35)}\)

\noindent \( P_2 = 1 - 1.37.10^{-4} z_m + 6.2.10^{-9} z_m^2 \)

\noindent \(A_3 =
\begin{cases}
  4.937.10^{-4} - 2.59.10^{-5} T + 9.11.10^{-7} T^2 - 1.5.10^{-8} T^3 & \text{for $T \leq 20$ }\\    
 3.964.10^{-4} - 1.146.10^{-5} T + 1.145.10^{-7} T^2 - 6.5.10^{-8} T^3 & \text{for $T \geq 20$ }\\  
\end{cases}
\)

\noindent \( P_3 = 1 - 3.83.10^{-5} z_m + 4.9.10^{-10} z_m^2 \)

\noindent with S in [ppt], f in [kHz], T in [$^{\circ}$C]. Furthermore $z_m$, ph and c denote the water depth in [m], the ph-value and the sound speed in [m/s] respectively.
