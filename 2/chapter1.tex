%DO NOT MESS AROUND WITH THE CODE ON THIS PAGE UNLESS YOU %REALLY KNOW WHAT YOU ARE DOING
 \chapter{Introduction}  \label{Introduction}



\noindent Salt water as a medium is dissipative. Due to viscosity of salt water (above 100 kHz), relaxation (i.e. the conversion of acoustic energy into heat) of magnesium sulphate (above 100 kHz or 10 kHz to 100 kHz) and boric acid (above 1 kHz or $\leq$10 kHz) sound waves are attenuated. There are numerous empirical models available for the prediction of attenuation of underwater sound, e.g. Thorp, Schulkin-Marsh and Francois-Garrison. The absorption of low frequency sound is comparably small. However, the magnitude of absorption increases with increasing frequency. In contrast to the sea surface, sound waves can penetrate into soft sea floors and consequently be absorbed significantly. Since the sea bed is a solid, also shear waves can develop. At long distances, losses in acoustic sound intensity result from attenuation and scattering losses. Sound attenuation in underwater environments is dependent on temperature, salinity, pressure and pH. If sound is propagating into the sea bed, it also is attenuated as a function of sediment type.

\noindent This assignment consists of three parts: In the first part we develop a MATLAB program that calculates the sound attenuation in seawater by means of Thorp, Schulkin \& Marsh and Francois \& Garrison formula. The results of the three approaches are then compared. The next two parts are done for Francois \& Garrison formula. The dependency on the frequency, salinity and temperature for a depth of 50m is investigated and finally a graph for attenuation versus frequency is plotted and frequency regions where the different attenuation processes dominate are specified.






