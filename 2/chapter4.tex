%DO NOT MESS AROUND WITH THE CODE ON THIS PAGE UNLESS YOU %REALLY KNOW WHAT YOU ARE DOING
\chapter{Conclusion} \label{Conclusion}

\noindent From the first part, it can be concluded that according to Thorp formula the attenuation of sound is increasing up to a certain range of frequency and then it remains unchanged. From Schulkin and Marsh formula we can state that attenuation of sound increases with increase in frequency. Francois and Garrison formula is almost similar to Schulkin and Marsh at higher frequencies and similar to Throp at low frequencies.

\noindent The contribution of boric acid, magnesium sulphate and pure water in the Francois-Garrison model was studied and it concludes that boric acid contributes to the overall contribution only at very low frequencies while pure water contributes at very high frequencies and magnesium sulphate in between the two.

\noindent Absorption decreases with temperature and it increases with frequency, salinity, and pH. At high
frequencies, the dependence of  attenuation coefficient on temperature is much stronger than that on the remaining frequencies. At low frequencies where the effect of boric acid is significant, absorption coefficient shows considerable variation with the pH of sea water. 

\noindent From the final set of graphs it can be proved that as the temperature increases there is increase in attenuation of sound, but after reaching a certain range of frequency there is slight change in attenuation which occurred due to the variations in salinity.
