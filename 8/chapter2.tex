%DO NOT MESS AROUND WITH THE CODE ON THIS PAGE UNLESS YOU %REALLY KNOW WHAT YOU ARE DOING
\chapter*{Theory}
\addcontentsline{toc}{chapter}{Theory}

\section{ Antenna Beam Pattern } \label{ Antenna Beam Pattern }
\noindent A radiation pattern or the antenna beam pattern defines the variation of the power radiated by an antenna as a function of the direction away from the antenna. Very often, only the relative amplitude is plotted, normalized either to the amplitude on the antenna boresight. As a consequence of the reciprocity theorem, the receiving pattern (sensitivity as a function of direction) is identical to the relative power density of the wave transmitted by the same antenna (power density as a function of direction). The pattern of an antenna may be determined experimentally at an antenna range, or alternatively, deduced by computation. The plots of antenna pattern can be used to benchmark a given radar antenna. They also tell you how much degradation you can expect if the antenna is not aimed properly.

\noindent The complex beam pattern for a line array is considered in this assignment. It is defined by
\begin{equation*}
 \tilde{b}(\beta) = \frac{1}{\hat{Q}} \sum_{n=0}^{N-1}Q_{n} e^{j{y_{n}^{'}} sin\beta}
 \end{equation*}
 \noindent with
  \begin{equation*}
 Q_{n} = \hat{Q_{n}} e^{j \alpha_{n}}
 \end{equation*}
 \noindent and
  \begin{equation*}
 \hat{Q} = \sum_{n=0}^{N-1}\hat{Q_{n}}
 \end{equation*}
\noindent For $Q_{0} = Q_{1} = ... = Q_{N-1}$, i.e., $\hat{Q_n} = 1$, $\alpha_n = 0$ and ${y_n}^{'} = -nd$, the complex beam pattern simplifies to
 \begin{equation*}
 \tilde{b}(\beta) = \frac{1}{N} \sum_{n=0}^{N-1} e^{jknd sin\beta}
 \end{equation*}
 \noindent where $k = 2\pi/\lambda$, d is the element spacing, n is the number of point sources and $\beta$ is the angle between vector r and axis x.
 
 \noindent The squared magnitude of complex beam pattern in dB is called beam pattern and it is expressed as
 \begin{equation*}
 \tilde{B}(\varphi,\theta) = 10\log_{10}{{|\tilde{b}(\varphi,\theta)|^{2}}} = 20\log_{10}{|\tilde{b}(\varphi,\theta)|}
  \end{equation*}

\noindent In the following sections, we will discuss in brief the parameters that alter the antenna beam pattern. The parameters discussed in the assignment are beam forming, amplitude shading, parabolic phase shading and linear phase shading.

\subsection{ Beam Forming } \label{ Beam Forming} 
\noindent Beam forming or spatial filtering is the process by which an array of large number of spatially separated sensors discriminate the signal arriving from a specified direction from a combination of isotropic random noise called ambient noise and other directional signals.

\subsection{ Amplitude Shading } \label{ Amplitude Shading } 
\noindent Shading is most commonly used to suppress side lobes (responses away from the main response lobe) or to suppress responses in noisy direction (known as Adaptive Beam forming). Having all coefficients equal, offers the maximum array gain in an isotropic noise field. Shading increases the width of the main lobe, decreases side lobes and reduces array gain. With shading, we are trading off the main lobe width and side lobe level.

\subsection{ Parabolic Phase Shading (Beam Shaping) } \label{ Parabolic Phase Shading (Beam Shaping) } 
\noindent According to our requirements, the beam pattern and main lobe can be designed this is known as beam shaping. Time delaying a signal is the time-domain analog to phase shading of the signal in frequency domain. The phase shading provides broadening of the main lobe.

\subsection{ Linear Phase Shading (Electronic Steering) } \label{ Linear Phase Shading (Electronic Steering) } 
\noindent Electronic steering is about changing the direction of the main lobe of a radiation pattern electronically by changing magnitude and phase of the hydrophone. Since there are no moving parts in Electronic Steering as compared to Mechanical steering, this method of beam steering is more efficient.

\noindent For $\alpha$ as an incident angle there would be a delay in receiving the wave by different elements, which corresponds to a phase shift given by
\begin{equation*}
 \varphi = knd sin(\alpha)
 \end{equation*}
 \begin{equation*}
 Q_{n} = e^{j\varphi} = e^{knd sin(\alpha)}
 \end{equation*}
 \noindent where \textit{$Q_{n}$} denote the amplitude of the $n^{th}$ point source with $n = 1,...,N.$  The complex beam pattern is defined as 
 \begin{equation*}
 \tilde{b}(\beta) = \frac{1}{\hat{Q}} \sum_{n=0}^{N-1}Q_{n} e^{j{y_{n}^{'}} sin\beta}
 \end{equation*}
 