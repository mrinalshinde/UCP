%DO NOT MESS AROUND WITH THE CODE ON THIS PAGE UNLESS YOU %REALLY KNOW WHAT YOU ARE DOING
\chapter*{Theory}
\addcontentsline{toc}{chapter}{Theory}

\section*{ Ambiguity Function } 
\addcontentsline{toc}{section}{Ambiguity Function}

\noindent The ambiguity is a two-dimensional function of delay and Doppler frequency showing the distortion of an uncompensated match filter due to the Doppler shift of the return from a moving target. It is useful for describing the behaviour of a radar or a sonar signal. This function appears to be a good tool to select good signals for the Range and Doppler estimation. It gives an idea of the amount of distortion present in the received signal. It is defined as the time response of a filter matched to a given finite energy signal when the signal is received with a delay $(\tau)$ and a doppler shift $(v)$ relative to the nominal values expected by the filter. The ambiguity function is defined by $-$
\begin{equation}
 \chi(\tau,v) = \int_{-\infty}^{\infty} s(t) s^{*}(t-\tau) e^{j2\pi vt}dt
 \end{equation}
 
 \noindent It can be interpreted as the output of a matched filter designed for a Doppler frequency shift $f$ if a signal with Doppler frequency shift $f_{0} +v$ is received. Thus $\chi(\tau,v)$ can be understood as the point target response in Range/Doppler domain.
\begin{equation} \label{eq2}
\begin{split}
\chi(\tau,v) & = \int_{-\infty}^{\infty} s(t) s^{*}(t-\tau) e^{j2\pi vt}dt \\
 & = \int_{-\infty}^{\infty} \Bigg( \int_{-\infty}^{\infty} S(\omega)e^{j\omega t}\frac{d\omega}{2\pi}\Bigg) \Bigg( \int_{-\infty}^{\infty} S^{*}(\omega^{'})e^{-j\omega^{'}(t-\tau)}\frac{d\omega^{'}}{2\pi}\Bigg) e^{j2\pi vt}dt \\
 & = \frac{1}{4\pi}\int_{-\infty}^{\infty}\int_{-\infty}^{\infty}\int_{-\infty}^{\infty} S(\omega)S^{*}(\omega{'})e^{j\omega^{'}\tau}e^{j(\omega - \omega^{'} + 2\pi v)t} d\omega d\omega{'} dt \\
 & = \frac{1}{4\pi}\int_{-\infty}^{\infty}\int_{-\infty}^{\infty}S(\omega)S^{*}(\omega{'})e^{j\omega^{'}\tau} \delta(\omega - \omega^{'} + 2\pi v) d\omega  d\omega{'} \\
 & = \frac{1}{4\pi}\int_{-\infty}^{\infty}S(\omega{'} - 2\pi v) S^{*}(\omega^{'})e^{j\omega^{'}\tau} d\omega^{'} \\
\end{split}
\end{equation}

\noindent The ambiguity function depends on the transmitted signal and one of the transmitted signals used here is a Linear Frequency Modulated pulse (LFM). In such a signal, the frequency is increased linearly resulting in a particular ambiguity function which is discussed in detail in the following sections.

\noindent Fast Fourier Transform (FFT) is an algorithm which is used to find the spectrum of any signal. FFT is a much faster algorithm as it reduces the number of computations involved in the fourier transform of the signal
\newpage
\section*{ Ambiguity Function of particular waveforms } 
\addcontentsline{toc}{section}{Ambiguity Function of particular waveforms}
\noindent The spectra of three waveforms are obtained using the FFT algorithm and they are studied. The three waveforms used are:
\begin{enumerate}
  \item A rectangular Pulse
  \item A Linear Frequency Modulated Pulse (LFM) with a rectangular envelope
  \item A Linear Frequency Modulated Pulse(LFM) with a Gaussian envelope
\end{enumerate}

\subsection*{ Rectangular Pulse } 
\addcontentsline{toc}{subsection}{Rectangular Pulse}
\noindent The simplest waveform for a radar system is probably a rectangular waveform, sometimes also referred to as single frequency waveform. For the rectangular waveform, the pulse width is the reciprocal of the bandwidth. The equation for the waveform is given by $-$
\begin{equation}
 s(t) = \frac{1}{\sqrt{T}}1_{(-T/2,T/2)}(t)
 \end{equation}
 \noindent with $||S||^{2} = 1$. Hence, the ambiguity function is given by
 \begin{equation} \label{eq4}
\begin{split}
\chi(\tau,v) & = \int_{-\infty}^{\infty} s(t) s^{*}(t-\tau) e^{j2\pi vt}dt \\
& =  \begin{cases} e^{j\pi v \tau} \Big( 1 - \frac{|\tau|}{T} \Big) \frac{sin(\pi v(T - |\tau|))}{\pi v(T - |\tau|)}, & \mbox{for $|\tau| \leq T$} \\ 0, & \mbox{elsewhere} \end{cases} \\
\end{split}
\end{equation}

\subsection*{ Linear Frequency Modulated Pulse (LFM) with a rectangular envelope } 
\addcontentsline{toc}{subsection}{Linear Frequency Modulated Pulse (LFM) with a rectangular envelope}
\noindent The equation for this waveform is given by $-$
\begin{equation}
 s(t) = \frac{1}{\sqrt{T}}1_{(-T/2,T/2)}(t) exp( j\pi k t^{2})
 \end{equation}
 \noindent with $k = b/T$. Hence, the ambiguity function is given by
 \begin{equation} \label{eq4}
\begin{split}
\chi(\tau,v) & = \int_{-\infty}^{\infty} s(t) s^{*}(t-\tau) e^{j2\pi vt}dt \\
& =  \begin{cases} e^{j\pi v \tau} \Big( 1 - \frac{|\tau|}{T} \Big) \frac{sin(\pi (k\tau + v)(T - |\tau|))}{\pi (k\tau + v)(T - |\tau|)}, & \mbox{for $|\tau| \leq T$} \\ 0, & \mbox{elsewhere} \end{cases} \\
\end{split}
\end{equation}
 
\subsubsection*{Determining the ambiguity function for a Linear Frequency Modulated Pulse (LFM) with a rectangular envelope} 
\begin{equation*} 
s(t) = e^{j(\pi k t^2)} \frac{1}{\sqrt{T}} rect(t) 
\end{equation*}
\begin{equation*} 
\chi(\tau,v) =  \frac{1}{\sqrt{T}} \frac{1}{\sqrt{T}} \int_{-\infty}^{\infty} rect(t) rect(t-\tau) e^{j(\pi kt^2)} e^{-j(\pi k {(t-\tau)}^2)}e^{j(2\pi kvt)} dt
\end{equation*}
\noindent Replacing: $ t = t^{'} + \frac{\tau}{2}$
\begin{equation*} 
\rightarrow \chi(\tau,v) =  \frac{1}{T} \int_{-\infty}^{\infty} rect(t^{'} + \frac{\tau}{2}) rect(t^{'} - \frac{\tau}{2}) e^{j(\pi k({t^{'} + \frac{\tau}{2}})^2)} e^{-j(\pi k {(t^{'} - \frac{\tau}{2})}^2)} e^{j(2\pi kv(t^{'}+\frac{\tau}{2}))} dt^{'}
\end{equation*}
\noindent Since ${(A+B)}^{2} - {(A-B)}^{2} = 4AB$
\begin{equation} 
\begin{split}
e^{j(\pi k({t^{'} + \frac{\tau}{2}})^2)}e^{-j(\pi k {(t^{'} - \frac{\tau}{2})}^2)} & = e^{j\Big(\pi k {(t^{'} + \frac{\tau}{2}})^{2} - {(t^{'} - \frac{\tau}{2}})^{2} \Big)}\\
& = e^{j(4\pi k t^{'} \frac{\tau}{2})} \\
& = e^{j(2\pi k t^{'} \tau)}
\end{split}
\end{equation}
\begin{equation*} 
\rightarrow \chi(\tau,v) =  \frac{1}{T} \int_{-\infty}^{\infty} rect(t^{'}+\frac{\tau}{2}) rect(t^{'}-\frac{\tau}{2}) e^{j(2\pi k t^{'} \tau)} e^{j(2\pi k t^{'})} dt^{'}
\end{equation*}
\begin{equation*} 
\rightarrow \chi(\tau,v) =  \frac{1}{T} \int_{-\infty}^{\infty} rect(t^{'}+\frac{\tau}{2}) rect(t^{'}-\frac{\tau}{2}) e^{j(2\pi (k\tau+v) t^{'})}  dt^{'}
\end{equation*}
\noindent Let $d(t) = \frac{T - |\tau|}{2}$
\begin{equation*} 
\rightarrow \chi(\tau,v) =  \frac{1}{T} \int_{-d(t)}^{d(t)} e^{j(2\pi (k\tau+v) t^{'})}  dt^{'}
\end{equation*}
\noindent Let $d(t) = \frac{T - |\tau|}{2}$
\begin{equation*} 
\rightarrow \chi(\tau,v) =  \frac{1}{T} e^{j(\pi v\tau)} \frac{e^{j(2\pi(k\pi + v)(d(t) - (-d(t))))}}{j(2\pi(k\tau + v)(d(t) - (-d(t))))}
\end{equation*}
\begin{equation*} 
\rightarrow \chi(\tau,v) =  \frac{1}{T} e^{j(\pi v\tau)} d(t^{'}) \frac{j2sin(2\pi(k\pi + v)(\frac{T-|\tau|}{2}))}{j(2\pi(k\tau + v)(\frac{T-|\tau|}{2}))}
\end{equation*}
\begin{equation*}
\chi(\tau,v) =  \begin{cases} e^{j\pi v \tau} \Big( 1 - \frac{|\tau|}{T} \Big) \frac{sin(\pi (k\tau + v)(T - |\tau|))}{\pi (k\tau + v)(T - |\tau|)}, & \mbox{for $|\tau| \leq T$} \\ 0, & \mbox{elsewhere} \end{cases} 
\end{equation*}
\noindent Hence proved

\subsection*{ Linear Frequency Modulated Pulse (LFM) with a Gaussian envelope } 
\addcontentsline{toc}{subsection}{Linear Frequency Modulated Pulse (LFM) with a Gaussian envelope}
\noindent The equation for this waveform is given by $-$
\begin{equation}
 s(t) = \frac{1}{\sqrt[4]{\pi \sigma ^{2}}} exp \Bigg( - \frac{t^2}{2\sigma ^{2}} + j\pi k t^2 \Bigg)
 \end{equation}
 \noindent where $k$ determines the slope of the LFM with $|k| = b/T$. After some manipulations, we obtain
\begin{equation} \label{eq6}
\begin{split}
\chi(\tau,v) & = \int_{-\infty}^{\infty} s(t) s^{*}(t-\tau) e^{j2\pi vt}dt \\
& =  e^{j\pi vt} exp \Big( -\tau ^{2} / (4 \sigma ^{2}) - \pi \sigma ^{2} {(k\tau + v)}^{2} \Big) \\
\end{split}
\end{equation}

 
