%DO NOT MESS AROUND WITH THE CODE ON THIS PAGE UNLESS YOU %REALLY KNOW WHAT YOU ARE DOING
\chapter{Theory} \label{Theory}

\section{ Sound Scattering } \label{ Sound Scattering }
\noindent Scattering results when sound strikes foreign bodies in the water, and the sound energy is reflected. Some reflectors are boundaries (surface, bottom, and shores), bubbles, suspended solid and organic particles, marine life, and minor inhomogeneities in the thermal structure of the ocean. The amount of energy scattered is a function of the size, density, and concentration of foreign bodies present in the sound path, as well as the frequency of the sound wave. The larger the area of the reflector compared to the sound wavelength, the more effective it is as a scatterer. Part of the reflected sound is returned to the source as an echo, i.e, is backscattered, and the remainder is reflected off in another direction and is lost energy. Back-scattered energy is known as reverberation and is divided into three types: surface, bottom and volume.

\begin{figure}[H]
\centering
{\includegraphics[scale=0.5]{scattering.png}}
\caption{ Different types of reverberations}
\end{figure}

\subsection{ Surface Backscattering } \label{ Surface Backscattering } 
\noindent Surface reverberation is generated when transmitted sound rays strike the surface of the ocean, i.e., the underside of the waves. It is always a factor in active sonar operations, and is directly related to wind speed because it controls wave size and the angle of incidence. The effectiveness of active sonar is severely restricted in shallow waters because in very shallow water, severe reverberation levels may exist due to the presence of so many surfaces for the sound to reflect off. Experiments indicate that the backscattering strength of the sea surface varies with the grazing angle, sound frequency and wind speed induced roughness and that the collected measurements can be fitted by the following empirical expression
\begin{align}
{S_S}  = 10 . log_{10}( 10^{-5.05}  . ( 1 + {v_W} )^2 . ( \textit{f} + 0.1 )^{\frac{v_W}{150}} . tan^{\beta}(\theta))
\end{align}
\noindent with

\begin{align}
\beta  = 4 . (\frac{v_W + 2}{v_W+ 1}) + ( 2.5 ( f + 0.1 )^{\frac{-1}{3}} - 4 ) . cos^{\frac{1}{8}}(\theta)
\end{align}
 
\noindent where $ {S_S} $ represents the surface backscattering coefficient in $ [dB/m^2] $. The parameters \textit{f}, ${v_W}$  and  $\theta$ denote the sound frequency in kHz, the wind speed in knots and the grazing angle respectively. 

\subsection{ Bottom Backscattering } \label{ Bottom Backscattering } 
\noindent Bottom reverberation occurs whenever a sound pulse strikes the ocean bottom. In deep water this condition normally does not cause serious problems, but in shallow water, bottom reverberation can dominate the background and completely mask a close target. 

\noindent Sound reflected from the ocean floor usually suffers a significant loss in intensity. Part of this loss is caused by the scattering effects just described, but most of it results from the fact that a portion of sound energy will enter the bottom and travel within it as a new wave. The net result is that the strength of the reflected wave is greatly reduced. The amount of energy lost into the bottom varies with the bottom composition, sound frequency, and the grazing angle of the sound wave. The total of these losses can vary from as low as 2 dB/bounce to greater than 30 dB/bounce. In general, bottom loss will tend to increase with frequency and with the angle of incidence. Soft bottoms such as mud are usually associated with high bottom losses (10 to 30 dB/bounce); hard bottoms such as smooth rock or sand produce lower losses.

\noindent Furthermore, it could be observed that a Lambert?s law relationship between the backscattering strength and the grazing angle fits to many experimental data satisfactorily accurate for angles below $60^{\circ}$. Consequently, the backscattering strength can be described by Lambert\textsc{\char13}s law and an empirically specified scattering coefficient, i.e.
\begin{align}
 {S_B}  = K ( \textit{f}, \textit{bt}) + 10 . log_{10}( sine^{2}(\theta)) 
 \end{align}
 \noindent This is valid for for $0 \leq \theta \leq 60^{\circ}$. K(\textit{f}, \textit{bt}) denotes the scattering coefficient depending on the frequency of sound \textit{f} and bottom type \textit{bt}.
 
\noindent Due to the empirical definition of K(\textit{f}, \textit{bt}) it is evident that one will have considerable difficulty in determining an appropriate value for the backscattering strength in practice. Therefore, easier applicable and over the entire grazing angle domain sufficient accurate bottom backscattering models are of interest.

\noindent More accurate bottom scattering curves have been derived from measurements taken from SEARAY Model which is applicable for the frequency range of $20 kHz \leq \textit{f} \leq 500 kHz$ and APL-UW Model which works for the frequency range of $1 kHz \leq \textit{f} \leq 500 kHz$

\noindent For the SEARAY Model the reverberation coefficient is defined by 
\begin{align}
{S_B}  = 10 . log_{10}( 3.03  \beta  \textit{ f } ^{ 3.2 - 0.8 \textit{bt} } 10^{2.8 \textit{bt} - 12} + 10^{-4.42})\end{align}
\noindent with
\begin{align}
\beta = \gamma (sin(\theta) + 0.19)^{\textit{bt}cos^{16}(\theta)}
\end{align}
\noindent and
\begin{align}
\gamma = 1 + 125 . exp( -2.64 . (\textit{bt}-1.75)^{2} - \frac{50}{\textit{bt}} . cot^{2}(\theta))
\end{align}
 \noindent where \textit{f}, \textit{bt} and $\theta$ denote the sound frequency in kHz, the bottom type and the grazing angle respectively. And \textit{bt} can be any real number satisfying $1 \leq \textit{bt} \leq 4.$


 \subsection{ Volume Backscattering } \label{ Volume Backscattering } 
 \noindent Volume reverberation is caused by various reflectors, but fish and other marine organisms are the major contributors. Additional causes are suspended solids, bubbles, and water masses of markedly different temperatures. Volume reverberation is always present during active sonar operations, but is not normally a serious factor in masking target echoes. The one exception involves the deep scattering layer (DSL), which is a relatively dense layer of marine life present in most areas of the ocean. During daylight hours, the layer is generally located at depths of about 600 meters and does not pose a serious problem. At night, however, the layer migrates toward the surface and becomes a major source of reverberation. 

\noindent The volume reverberation can be modeled by the so-called volume reverberation coefficient.
\begin{align}
{S_V} = {S_p} + 7 log_{10}(\textit{f})
\end{align}
\noindent where \textit{f} and ${S_p}$indicate the frequency in kHz and the particle contribution in $[dB/m^3].$ The particle contribution parameter ${S_p}$ is $-50 dB$ for high particle density, $-70 dB$ for moderate particle density and $-90 dB$ for low particle density.


