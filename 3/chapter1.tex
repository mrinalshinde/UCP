%DO NOT MESS AROUND WITH THE CODE ON THIS PAGE UNLESS YOU %REALLY KNOW WHAT YOU ARE DOING
 \chapter{Introduction}  \label{Introduction}

\noindent When an active sonar pulse is transmitted into the water, some of the sound reflects off of
the target. Additionally, there are many other sources where the sound energy may reradiate a portion of the acoustic energy incident upon them. This scattering is caused by the many sources of inhomogeneities in the ocean. These sources may include fish,  air bubbles, dust or dirt as well as the ocean bottom, and surface. The above scatterers can be classified into three basically different classes depending on the reverberation they produce. Scatterers occurring in the volume or body of the sea produce volume reverberation. Scatterers on or near the surface cause surface reverberation and scatterers on or near the sea bottom produce bottom reverberation. 

\noindent In this assignment we first develop a MATLAB program for computing the surface, bottom and volume reverberation coefficient. Then we plot the surface and bottom reverberation coefficients versus the grazing angle for various sets of (frequency, windspeed) and (frequency, bottom type) respectively. Finally the graph for volume reverberation versus frequency for high, moderate and low particle densities is obtained.





