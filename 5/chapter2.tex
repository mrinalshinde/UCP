%DO NOT MESS AROUND WITH THE CODE ON THIS PAGE UNLESS YOU %REALLY KNOW WHAT YOU ARE DOING
\chapter*{Theory}
\addcontentsline{toc}{chapter}{Theory}

\noindent The ocean is filled with different kinds of sounds. Underwater sound is generated by a variety of natural sources, such as breaking waves, wind, rain, and marine life. It is also generated by a variety of man-made sources, such as ships and sonars. This background sound in the ocean is called ambient noise.
The isotropic noise level consists of the following components
\begin{enumerate}
  \item Turbulence noise \\
\noindent This can be a significant factor in ambient noise levels at very low frequencies, between 1Hz to 10 Hz. Equation 1 is a function of frequency, \textit{f} which is in kHz.
\begin{equation}
 NL_{turb}(\textit{f}) = 30 - 30 . log_{10}(\textit{f}) 
\end{equation}

  \item Far shipping (traffic) noise \\
\noindent  There can be more than 1000 ships underway at any one time. The noise from this shipping traffic can sometimes travel up to distances of 1000 miles or more. The frequency range where this man-made noise is most dominant is from 10 Hz to 300 Hz. Noise levels depend on area operating in and shipping density. Close proximity to shipping lanes and harbors increases noise levels. Equation 2 is a function of frequency, \textit{f} which is in kHz.
\begin{equation}
 NL_{traffic}(\textit{f}) = 10 . log_{10}(\frac{3 . 10^8}{ 1 + 10^{4} . \textit{f} ^4 }) 
\end{equation}

  \item Sea state noise \\
\noindent Sea State (or more importantly wind speed) is the dominant factor in calculating ambient
noise levels between 300 Hz to 100 kHz. The noise levels depend on wind speed, $v_W$ in kn and frequency, \textit{f} in kHz.
\begin{equation}
 NL_{ss}(\textit{f}) = 40 + 10 . log_{10}(\frac{{v_w}^2}{ 1 + \textit{f} ^{\frac{5}{3}}}) 
\end{equation}

  \item Thermal noise \\
\noindent At frequencies between 100 kHz and 1 MHz, the noise generated by the random motion of water molecules is called thermal noise. This noise depends upon frequency, \textit{f} which is in kHz.
\begin{equation}
 NL_{therm}(\textit{f}) = -15 + 20 . log_{10}(\textit{f}) 
\end{equation}

  \item Rainfall noise \\
\noindent It is dominant at frequencies between 1 kHz to 5 kHz. It is denoted by $NL_{rain}(\textit{f}, \textit{r})$ where \textit{f} and \textit{r} denote the frequency and rainfall rate, respectively.

\item Biological noise \\
\noindent Noise produced by snapping shrimp and other fishes typically comprises of biological noise. It is denoted by $NL_{bio}(\textit{f}, \textit{s})$ where \textit{f} and \textit{s} denote the frequency and season, respectively.

\item Self (vessel) noise of sonar platform \\
\noindent It is denoted by $NL_{vessel}(\textit{f}, {v_V})$ where \textit{f} and ${v_V}$ denote the frequency and vessel speed, respectively.
 \end{enumerate}

\noindent  Thus the isotropic noise level can be determined by

\begin{equation}
\begin{aligned}[b]
 & NL (\textit{f}, {v_W},\textit{r},\textit{s}, {v_V}) = 10 . log_{10}(10^{0.1 . NL_{turb}} + 10^{0.1 . NL_{traffic}} + 10^{0.1 . NL_{ss}} \\
 & + 10^{0.1 . NL_{therm}} + 10^{0.1 . NL_{rain}} + 10^{0.1 . NL_{bio}} + 10^{0.1 . NL_{vessel}} )
 \end{aligned}
\end{equation}




