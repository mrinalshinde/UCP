%DO NOT MESS AROUND WITH THE CODE ON THIS PAGE UNLESS YOU %REALLY KNOW WHAT YOU ARE DOING
\chapter{Theory} \label{Theory}

\section{ Sonar performance prediction } \label{ Sonar performance prediction }

\noindent The basic problem in sonar is to measure some signal (possibly an echo) against a background of noise (or reverberation). So that the signal may be detected above the background, the ratio of the measured signal to the measured background (signal-to-noise ratio) must be at least some minimum value that is determined by the system. The various terms in the sonar equations are called sonar parameters. These parameters may be grouped according to whether they are determined by the equipment, medium, or target. Equipment parameters are source level, detection threshold, directivity index, self-noise level, and array gain (also determined by medium). Medium parameters are transmission loss, ambient noise level, and reverberation level (also determined by equipment). And target parameters are target strength, and target source level. Oceanographic interest in marine acoustics is concentrated mostly in the parameters determined by the medium.

\noindent Sonar may be either active or passive. Active sonar provide their own sound source and listen for echoes as they are reflected from the target which they are trying to detect. Passive sonar have no such source. They are simply listening devises that rely on the target to emit their own noise source (e.g. engine noise from an enemy ship or communication noises from a whale).

\section{ Sound propagation related parameters } \label{ Sound propagation related parameters }

\noindent The parameters described in the following sections are transmission loss (\textit{TL}), isotropic noise level (\textit{NL}), bottom reverberation strength (\textit{$RS_B$}), surface reverberation strength (\textit{$RS_S$}) and volume reverberation strength (\textit{$RS_V$}).

\subsection{ Transmission loss } \label{ Transmission loss }
\noindent The intensity of an acoustic signal reduces with range. This observed reduction in the acoustic signal with distance from the source is due to the combined effects of spreading and attenuation and is accounted for by the transmission loss term.



