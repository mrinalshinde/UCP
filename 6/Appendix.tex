%DO NOT MESS AROUND WITH THE CODE ON THIS PAGE UNLESS YOU %REALLY KNOW WHAT YOU ARE DOIN
\chapter*{Appendix}
\addcontentsline{toc}{chapter}{Appendix}

\section{ MATLAB functions required to plot dependence of bottom type and wind speed on signal to noise ratio } \label{ MATLAB functions required to plot dependence of bottom type and wind speed on signal to noise ratio }

\subsection{ MATLAB function to compute Transmission Loss} \label{ MATLAB program to compute transmission loss } 
\lstset{language=Matlab,%
    %basicstyle=\color{red},
    basicstyle=\tiny,
    breaklines=true,%
    morekeywords={matlab2tikz},
    keywordstyle=\color{blue},%
    morekeywords=[2]{1}, keywordstyle=[2]{\color{black}},
    identifierstyle=\color{black},%
    stringstyle=\color{mylilas},
    commentstyle=\color{mygreen},%
    showstringspaces=false,%without this there will be a symbol in the places where there is a space
    numbers=left,%
    numberstyle={\tiny \color{black}},% size of the numbers
    numbersep=9pt, % this defines how far the numbers are from the text
    emph=[1]{for,end,break},emphstyle=[1]\color{red}, %some words to emphasise
    %emph=[2]{word1,word2}, emphstyle=[2]{style},    
}
\lstinputlisting{Atten_Schulkin_Marsh.m}

\subsection{ MATLAB function to compute the noise level} \label{ MATLAB function to compute the noise level} 
\lstinputlisting{Noise_Level.m}

\subsection{ MATLAB function to compute the bottom reverberation} \label{ MATLAB function to compute the bottom reverberation} 
\lstinputlisting{Bottom_Reverberation.m}

\subsection{ MATLAB function to compute the surface reverberation} \label{ MATLAB function to compute the surface reverberation} 
\lstinputlisting{Surface_Reverberation.m}

\subsection{ MATLAB function to compute the volume reverberation} \label{ MATLAB function to compute the volume reverberation}
\lstinputlisting{Volume_Reverberation.m}

\section{ MATLAB program to plot dependence of bottom type and wind speed on signal to noise ratio } \label{ MATLAB program to plot dependence of bottom type and wind speed on signal to noise ratio }
\lstinputlisting{USP5.m}



