%DO NOT MESS AROUND WITH THE CODE ON THIS PAGE UNLESS YOU %REALLY KNOW WHAT YOU ARE DOING
 \chapter*{Introduction}
\addcontentsline{toc}{chapter}{Introduction}

\noindent The sonar equation is a systematic way of estimating the expected signal-to-noise ratios for sonar systems. The signal-to-noise ratio determines whether or not a sonar will be able to detect a signal in the presence of background noise in the ocean. It takes into account the source level, sound spreading, sound absorption, reflection losses, ambient noise, and receiver characteristics. The sonar equation is used to estimate the expected signal-to-noise ratios for all types of sonar systems.  

\noindent In this assignment we first develop a MATLAB program for determining the signal to noise ratio. The calculations for various parameters are then carried out. We then discuss the impact of bottom type (mud, sand and gravel) and wind speed on the signal to noise ratio.



