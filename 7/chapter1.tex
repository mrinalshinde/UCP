%DO NOT MESS AROUND WITH THE CODE ON THIS PAGE UNLESS YOU %REALLY KNOW WHAT YOU ARE DOING
 \chapter*{Introduction}
\addcontentsline{toc}{chapter}{Introduction}

\noindent For directional reception of wave energy, it is necessary to use equidistantly arranged linear or planar arrays on extended antennas. The sonar transmitting/receiving antenna characteristics can be determined by the geometry and shading of the antenna aperture and also the properties of individual transducers. If simultaneous detection of signals from many directions of incidence is required, beam forming must be carried out in parallel
for many direction channels. It is also possible to steer antenna in defined direction and range zones and thus process near-field signals selectively with regard to bearing and range of their source.


\noindent In this assignment we first develop a MATLAB program that determines the beam pattern. Then the graphs are plotted for beam pattern using various parameters. The parameters used in this assignment are beam forming, amplitude shading, beam shaping and electronic sheering.


