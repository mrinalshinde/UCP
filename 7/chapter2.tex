%DO NOT MESS AROUND WITH THE CODE ON THIS PAGE UNLESS YOU %REALLY KNOW WHAT YOU ARE DOING
\chapter*{Theory}
\addcontentsline{toc}{chapter}{Theory}

\section{ Antenna Beam Pattern } \label{ Antenna Beam Pattern }
\noindent

\subsection{ Beam Forming } \label{ Beam Forming} 
\noindent Beam forming or spatial filtering is the process by which an array of large number of spatially separated sensors discriminate the signal arriving from a specified direction from a combination of isotropic random noise called ambient noise and other directional signals.

\subsection{ Amplitude Shading } \label{ Amplitude Shading } 
\noindent Shading is most commonly used to suppress side lobes (responses away from the main response lobe) or to suppress responses in noisy direction (known as Adaptive Beam forming). Having all coefficients equal offers the maximum array gain in an isotropic noise field. Shading increases, the width of the main lobe and decreases side lobes and reduces array gain. With the shading we are trading off the main lobe width and side lobe level.

\subsection{ Parabolic Phase Shading (Beam Shaping) } \label{ Parabolic Phase Shading (Beam Shaping) } 
\noindent According to our requirements, the beam pattern and main lobe can be designed this is known as beam shaping. Time delaying a signal is the time-domain analog to phase shading of the signal in frequency domain.

\subsection{ Linear Phase Shading (Electronic Steering) } \label{ Linear Phase Shading (Electronic Steering) } 
\noindent Electronic steering is about changing the direction of the main lobe of a radiation pattern electronically by changing magnitude and phase of the hydrophone. Since there are no moving parts in Electronic Steering as compared to Mechanical steering, this method of beam steering is more efficient.


