%DO NOT MESS AROUND WITH THE CODE ON THIS PAGE UNLESS YOU %REALLY KNOW WHAT YOU ARE DOING
\chapter{Theory} \label{Theory}
\section{Sound velocity in ocean} \label{Sound velocity in ocean}
\noindent Variations of the sound velocity in the ocean are relatively small due to low absorption compared to speed of light in water. Blue-green light has the lowest attenuation in water upto 20m while sound can travel upto 100km one way and hence sonar is used to navigate, communicate with or detect objects on or under the surface of the water. Sound velocity in ocean lies between about 1450 m/s and 1540 m/s. Even though the changes of sound speed are small, i.e.  $\leq$ $\pm 3$\%, the propagation of sound can be significantly affected.

\noindent The sound velocity can be directly measured by velocimeters or calculated by empirical formulae if the temperature, salinity and hydrostatic power or depth are known. However, the error of measurements by modern velocimeters is about 0.1 m/s. The accuracy of calculations by the most complete empirical formulae is about the same. Also formulae providing such high accuracy are very cumbersome. 

\noindent A less accurate but simpler equation is given by:

c = 1449.2 + 4.6T $-$ $0.055T^2$ + $0.00029T^3$ + $(1.34 -0.01T) (S-35)$ + 0.016z \\
\noindent where temperature, T in ($^{\circ}C$), salinity, S in (ppt), depth, z in (m) and sound velocity, c in (m/s) 

\noindent With the following limits: \\
\noindent 0$^{\circ}C$   $\leq$ T   $\leq$ 35$^{\circ}$C  \\
\noindent 0 ppt   $\leq$S   $\leq$45 ppt \\
\noindent 0 m   $\leq$ z   $\leq$1000 m

\noindent The function for velocity of sound in ocean and the dependency of nonlinear temperature profile and the depth of the ocean is studied. Two sets of waveforms of c versus z for various sets of T and S are plotted in-order to investigate the dependence of c on T, S and z. Also the waveform for the nonlinear temperature profile and the depth of the ocean is obtained and discussed in the following sections.